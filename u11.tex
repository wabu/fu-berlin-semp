\documentclass[a4paper,10pt]{report}

\topmargin -2cm
%\topskip0cm
%\footskip0cm
%\headsep0cm
\parindent0cm
\oddsidemargin -1cm
\evensidemargin -1cm
\headheight 2cm
\textheight 24cm
\textwidth 18cm

\author{Daniel W\"aber (4049590)}
\title{\"Ubung}

\usepackage{ucs}
\usepackage[utf8x]{inputenc}
\usepackage{german}
\usepackage{color}
\usepackage{url}
\usepackage{graphicx}
\usepackage{algorithmic}

\pagestyle{empty}
\usepackage{makeidx}
\usepackage{amsmath}
\usepackage{amsfonts}
\usepackage{amssymb,euscript}
\usepackage{dsfont}
\usepackage{listings}
\usepackage{enumerate}
\newfont{\Fr}{eufm10}
\newfont{\Sc}{eusm10}
\newfont{\Bb}{msbm10}
\newcommand{\limin}{\lim_{n\rightarrow\infty}}
\newcommand{\limix}{\lim_{x\rightarrow\infty}}
\newcommand{\limun}{\lim_{n\rightarrow -\infty}}
\newcommand{\limux}{\lim_{n\rightarrow -\infty}}
\newcommand{\limx}{\lim_{x\rightarrow x_0}}
\newcommand{\limh}{\lim_{h\rightarrow 0}}
\newcommand{\defi}{\paragraph{Definition:}}
\newcommand{\bew}{\paragraph{Beweis:}}
\newcommand{\satz}{\paragraph{Satz:}}
\newcommand{\bsp}{\paragraph{Beispiel:}}
\newcommand{\lemma}{\paragraph{Lemma:}}
\newcommand{\N}{\mathds{N}}
\newcommand{\F}{\mathds{F}}
\newcommand{\Z}{\mathds{Z}}
\newcommand{\Q}{\mathds{Q}}
\newcommand{\R}{\mathds{R}}
\newcommand{\G}{\mathds{G}}
\newcommand{\C}{\mathds{C}}
\newcommand{\K}{\mathds{K}}
\newcommand{\A}{\mathds{A}}
\newcommand{\E}{\mathcal{E}}
\renewcommand{\P}{\mathcal{P}}
\newcommand{\sigA}{$\sigma$-Algebra }
\newcommand{\qed}{$\hfill\blacksquare$}
\newcommand{\arsinh}{\operatorname{arsinh} }
\newcommand{\arcosh}{\operatorname{arcosh} }
\newcommand{\gdw}{ $ \Leftrightarrow $ }
\newcommand{\tf}{ $ \Rightarrow $ }
\newcommand{\mgdw}{\Leftrightarrow}
\newcommand{\mtf}{\Rightarrow}
\newcommand{\Bild}{\text{Bild}}
\newcommand{\Kern}{\text{kern}}
\newcommand{\rg}{\text{rg}}
\newcommand{\deff}{\text{deff}}

\newcommand{\alphato}{\underset{\alpha}\to}
\newcommand{\betato}{\underset{\beta}\to}
\newcommand{\etato}{\underset{\eta}\to}
\newcommand{\ito}{\underset{i}\to}
\newcommand{\sto}{\underset{s}\to}
\newcommand{\kto}{\underset{k}\to}
\newcommand{\xto}{\underset{x}\to}

\usepackage{fancyhdr}
\pagestyle{fancy}
\lhead{Daniel Waeber\\<+Patner+>}
\chead{"Ubungsblatt \nr\\\today}
\rhead{<+Fach+>\\Tutor: <+Tutor+>}



\newcommand{\nr}{11}

\begin{document}
\section*{Aufgabe 1}
\begin{verbatim}
z = read;
n = read;
c = 0;
while z>0 do
    c = c+1;
    z = z-n;
output n
\end{verbatim}

\newpage

\section*{Aufgabe 2}
\begin{equation}
\C{\text{for}(C_1, B, C_2) C} = \C{C_1} (\B{ B } \text{cond}\langle \C{C; C_2; \text{for}(\text{skip},B,C_2)C} , id\rangle )
\end{equation}
oder
\begin{equation}
\C{\text{for }I = T_1 \text{ to } T_2 \text{ do } C} = 
\T{T_1} \lambda n_1 . \T{T_2} \lambda n_2 \lambda \langle s,e,a\rangle . \text{cond}\langle \C{C ; \text{for }I = n_1+1 \text{ to } n_2 \text{ do } C} \langle s[n_1/I],e,a\rangle, \text{id}\rangle (n_1\leq n_2)
\end{equation}

\section*{Aufgabe 3}
Da bei eine Fehler direkt abgebrochen werden kann, muss dieser nicht durch mehrere Funktionen geschleift werden.

\section*{Aufgabe 4}
$( (\lambda z. z) * \pi_3 )$ ist vom Typ $ \text{Zustand} \to \text{Ausgabe}+\text{Fehler}_\bot)$, also entspricht nicht dem Typ Fortsetzung.
$\pi_3$ darf als erst spaeter angewandt werden.

\section*{Aufgabe 5}
Siehe \url{http://www.inf.fu-berlin.de/lehre/SS10/Semantik/Aufgaben/while.hs}

\end{document}
