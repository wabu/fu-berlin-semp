\documentclass[a4paper,10pt]{report}

\topmargin -2cm
%\topskip0cm
%\footskip0cm
%\headsep0cm
\parindent0cm
\oddsidemargin -1cm
\evensidemargin -1cm
\headheight 2cm
\textheight 24cm
\textwidth 18cm

\author{Daniel W\"aber (4049590)}
\title{\"Ubung}

\include{header}
\usepackage{fancyhdr}
\pagestyle{fancy}
\lhead{Daniel Waeber\\Alex Muenn}
\chead{"Ubungsblatt \nr\\\today}
\rhead{Semantik von Programiersprachen}



\newcommand{\nr}{11}

\begin{document}
\section*{Aufgabe 1}
\begin{verbatim}
z = read;
n = read;
c = 0;
while z>0 do
    c = c+1;
    z = z-n;
output n
\end{verbatim}

\newpage

\section*{Aufgabe 2}
\begin{equation}
\C{\text{for}(C_1, B, C_2) C} = \C{C_1} (\B{ B } \text{cond}\langle \C{C; C_2; \text{for}(\text{skip},B,C_2)C} , id\rangle )
\end{equation}
oder
\begin{equation}
\C{\text{for }I = T_1 \text{ to } T_2 \text{ do } C} = 
\T{T_1} \lambda n_1 . \T{T_2} \lambda n_2 \lambda \langle s,e,a\rangle . \text{cond}\langle \C{C ; \text{for }I = n_1+1 \text{ to } n_2 \text{ do } C} \langle s[n_1/I],e,a\rangle, \text{id}\rangle (n_1\leq n_2)
\end{equation}

\section*{Aufgabe 3}
Da bei eine Fehler direkt abgebrochen werden kann, muss dieser nicht durch mehrere Funktionen geschleift werden.

\section*{Aufgabe 4}
$( (\lambda z. z) * \pi_3 )$ ist vom Typ $ \text{Zustand} \to \text{Ausgabe}+\text{Fehler}_\bot)$, also entspricht nicht dem Typ Fortsetzung.
$\pi_3$ darf als erst spaeter angewandt werden.

\section*{Aufgabe 5}
Siehe \url{http://www.inf.fu-berlin.de/lehre/SS10/Semantik/Aufgaben/while.hs}

\end{document}
