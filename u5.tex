\documentclass[a4paper,10pt]{report}

\topmargin -2cm
%\topskip0cm
%\footskip0cm
%\headsep0cm
\parindent0cm
\oddsidemargin -1cm
\evensidemargin -1cm
\headheight 2cm
\textheight 24cm
\textwidth 18cm

\author{Daniel W\"aber (4049590)}
\title{\"Ubung}

\include{header}
\usepackage{fancyhdr}
\pagestyle{fancy}
\lhead{Daniel Waeber\\Alex Muenn}
\chead{"Ubungsblatt \nr\\\today}
\rhead{Semantik von Programiersprachen}



\newcommand{\nr}{5}

\begin{document}
\section*{Aufgabe 1}
$\C{\text{read }I} z = 
\begin{cases}
    \text{Fehler} & \text{falls $e=\varepsilon$ oder $e=b.e'$} 
\\
    (s[n/I], e', a) & \text{falls $e=n.e'$}
\end{cases}$

\section*{Aufgabe 2}
$\C{\text{for } I:=T \text{ to } N \text{ do } C} z = 
\begin{cases}
    \C{\bar{C}} z' & \text{falls $T<N$,
                           wobei $z' = \C{C}(s[T/I],e,a)$,
                           $\bar{C}=\text{for } I:=\underline{T+1} \text{ to } N \text{ do }$
                        } 
\\
    z & \text{falls $T=N$} 
\\
    \text{Fehler} & \text{sonst}
\end{cases}$

\section*{Aufgabe 3}

$\B{\text{eof}} z = \begin{cases}
    (\text{wahr},z) & \text{falls $z = (s,\varepsilon,a)$} \\
    (\text{falsch},z) & \text{sonst}
\end{cases}$

\section*{Aufgabe 4}

aus $A.5$:
$$\frac
    { \{true\} x:=7 \{x=7\}, \{x=7\} y:=x+3 \{y=10\} }
    { \{true\} x:=7; y:=x+3 \{y=10\} }
$$

$\{true\} x:=7 \{x=7\}$ und $\{x=7\} y:=x+3 \{y=10\}$ gelten nach $A.2$

\section*{Aufgabe 6}

\end{document}
