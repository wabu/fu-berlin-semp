\documentclass[a4paper,10pt]{report}

\topmargin -2cm
%\topskip0cm
%\footskip0cm
%\headsep0cm
\parindent0cm
\oddsidemargin -1cm
\evensidemargin -1cm
\headheight 2cm
\textheight 24cm
\textwidth 18cm

\author{Daniel W\"aber (4049590)}
\title{\"Ubung}

\usepackage{ucs}
\usepackage[utf8x]{inputenc}
\usepackage{german}
\usepackage{color}
\usepackage{url}
\usepackage{graphicx}
\usepackage{algorithmic}

\pagestyle{empty}
\usepackage{makeidx}
\usepackage{amsmath}
\usepackage{amsfonts}
\usepackage{amssymb,euscript}
\usepackage{dsfont}
\usepackage{listings}
\usepackage{enumerate}
\newfont{\Fr}{eufm10}
\newfont{\Sc}{eusm10}
\newfont{\Bb}{msbm10}
\newcommand{\limin}{\lim_{n\rightarrow\infty}}
\newcommand{\limix}{\lim_{x\rightarrow\infty}}
\newcommand{\limun}{\lim_{n\rightarrow -\infty}}
\newcommand{\limux}{\lim_{n\rightarrow -\infty}}
\newcommand{\limx}{\lim_{x\rightarrow x_0}}
\newcommand{\limh}{\lim_{h\rightarrow 0}}
\newcommand{\defi}{\paragraph{Definition:}}
\newcommand{\bew}{\paragraph{Beweis:}}
\newcommand{\satz}{\paragraph{Satz:}}
\newcommand{\bsp}{\paragraph{Beispiel:}}
\newcommand{\lemma}{\paragraph{Lemma:}}
\newcommand{\N}{\mathds{N}}
\newcommand{\F}{\mathds{F}}
\newcommand{\Z}{\mathds{Z}}
\newcommand{\Q}{\mathds{Q}}
\newcommand{\R}{\mathds{R}}
\newcommand{\G}{\mathds{G}}
\newcommand{\C}{\mathds{C}}
\newcommand{\K}{\mathds{K}}
\newcommand{\A}{\mathds{A}}
\newcommand{\E}{\mathcal{E}}
\renewcommand{\P}{\mathcal{P}}
\newcommand{\sigA}{$\sigma$-Algebra }
\newcommand{\qed}{$\hfill\blacksquare$}
\newcommand{\arsinh}{\operatorname{arsinh} }
\newcommand{\arcosh}{\operatorname{arcosh} }
\newcommand{\gdw}{ $ \Leftrightarrow $ }
\newcommand{\tf}{ $ \Rightarrow $ }
\newcommand{\mgdw}{\Leftrightarrow}
\newcommand{\mtf}{\Rightarrow}
\newcommand{\Bild}{\text{Bild}}
\newcommand{\Kern}{\text{kern}}
\newcommand{\rg}{\text{rg}}
\newcommand{\deff}{\text{deff}}

\newcommand{\alphato}{\underset{\alpha}\to}
\newcommand{\betato}{\underset{\beta}\to}
\newcommand{\etato}{\underset{\eta}\to}
\newcommand{\ito}{\underset{i}\to}
\newcommand{\sto}{\underset{s}\to}
\newcommand{\kto}{\underset{k}\to}
\newcommand{\xto}{\underset{x}\to}

\usepackage{fancyhdr}
\pagestyle{fancy}
\lhead{Daniel Waeber\\<+Patner+>}
\chead{"Ubungsblatt \nr\\\today}
\rhead{<+Fach+>\\Tutor: <+Tutor+>}



\newcommand{\nr}{5}

\begin{document}
\section*{Aufgabe 1}
$\C{\text{read }I} z = 
\begin{cases}
    \text{Fehler} & \text{falls $e=\varepsilon$ oder $e=b.e'$} 
\\
    (s[n/I], e', a) & \text{falls $e=n.e'$}
\end{cases}$

\section*{Aufgabe 2}
$\C{\text{for } I:=T \text{ to } N \text{ do } C} z = 
\begin{cases}
    \C{\bar{C}} z' & \text{falls $T<N$,
                           wobei $z' = \C{C}(s[T/I],e,a)$,
                           $\bar{C}=\text{for } I:=\underline{T+1} \text{ to } N \text{ do }$
                        } 
\\
    z & \text{falls $T=N$} 
\\
    \text{Fehler} & \text{sonst}
\end{cases}$

\section*{Aufgabe 3}

$\B{\text{eof}} z = \begin{cases}
    (\text{wahr},z) & \text{falls $z = (s,\varepsilon,a)$} \\
    (\text{falsch},z) & \text{sonst}
\end{cases}$

\section*{Aufgabe 4}

aus $A.5$:
$$\frac
    { \{true\} x:=7 \{x=7\}, \{x=7\} y:=x+3 \{y=10\} }
    { \{true\} x:=7; y:=x+3 \{y=10\} }
$$

$\{true\} x:=7 \{x=7\}$ und $\{x=7\} y:=x+3 \{y=10\}$ gelten nach $A.2$

\section*{Aufgabe 6}

\end{document}
