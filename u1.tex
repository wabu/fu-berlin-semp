\documentclass[a4paper,10pt]{report}

\topmargin -2cm
%\topskip0cm
%\footskip0cm
%\headsep0cm
\parindent0cm
\oddsidemargin -1cm
\evensidemargin -1cm
\headheight 2cm
\textheight 24cm
\textwidth 18cm

\author{Daniel W\"aber (4049590)}
\title{\"Ubung}

\include{header}
\usepackage{fancyhdr}
\pagestyle{fancy}
\lhead{Daniel Waeber\\Alex Muenn}
\chead{"Ubungsblatt \nr\\\today}
\rhead{Semantik von Programiersprachen}



\newcommand{\nr}{1}

\begin{document}
\section*{Aufgabe 1}
\begin{eqnarray}
N &:==& 0 | 1 | 2 | \cdots | 9 \\
Z &:==& N | -1 | -2 | \cdots \\
R &:==& Z . N \\
W &:==& \text{true} | \text{false} \\
K &:==& Z | R | W \\
I &:==& a | b | \cdots | Z | a_1 | a_2 | \cdots \\
OP &:==& + | - | * | ÷ | \mod \\
BOP &:==& = | < | > | \leq | \cdots \\
T &:==& Z | R | I | T_1 OP T_2 | \text{read} \\
B &:==& W | \text{not $B$} | \text{$T_1$ $BOP$ $T_2$}  | \text{read}\\
C &:==& \text{skip} | I:=T | \text{$C_1$; $C_2$} | \text{if $B$ then $C_1$ else $C_2$} | 
        \text{while $B$ do $C$} | \text{output $T$} | \text{output $B$}
\end{eqnarray}

\section*{Aufgabe 2}
<+ TODO: was is gemeint +>

\section*{Aufgabe 3}
<+ TODO: ist das so gemeint+>
\begin{itemize}
\item bei $Z, R$ und $W$ werden als entsprechende zahlen ausgewertet, wobei true 1 und false 0 ist
\item fuer jeden Bezeichner im programm wird eine Speicheradresse festgelegt
\item bei $I:=T$ wird der speicheradersse von $I$ das Ergebnis von $T$ zugeordnet
\item das Ergebnis von $I$ ist der Wert an der Spechererstelle von $I$
\item das Ergebnis von $T_1\>OP\>T>2$ wird die Operation auf das Ergebnis von $T_1$ und $T_2$ angewandt
    \\ im falle einer Division von 0 oder einse Ueberlaufs bricht das Program mit einem Fehler ab
\item read liest entsprechend der umgebung zahlen oder boolsche ausdrucke
    \\ ist die eingabedatei leer, bricht das programm mit einem Fehler ab
    \\ kann die eingabe nicht in eine Zahl umgeformt werden, oder wird ein boolscher ausdruck in einem read einer Zahl eingelesen,
        bricht das programm mit einem Typfehler ab
\end{itemize}

\end{document}
